\section{Креирање ,,веб-паука''}

\subsection{Смештање хипервеза у листу}

Користећи претходно описано проналажење свих хипервеза(в. к\^{о}д \ref{lst:print_all_links}), а за потребе креирања веб-паука, потребно је сакупити све хипервезе у неку колекцију, како би касније ти подаци могли бити даље процесуирани.

У наредном к\^{о}ду, све хипервезе ће бити скупљене у листу. Уместо листе, могуће је користити и друге структуре као на пример мапе, али такви примери ће бити реализовани касније у оквиру овог рада.

\begin{lstlisting}[caption=Процедура смештања свих хиперлинкова у листу, label={lst:getalllinks}, numbers=left]
def get_all_links(page):
    links = []
    while True:
        url, endpos = get_next_target(page)
        if url:
            links.append(url)
            page=page[endpos:]
        else:
            break
    return links
\end{lstlisting}

У линији 4 се користе уређене n-торке, додељују се вредности променљивима \textbf{url} и \textbf{endpos}. Ако хипервеза постоји, биће додата у листу и траже се друге хипервезе почев од последњег знака наводника претходне хипервезе. Иначе, ако линк не постоји, петља се прекида и процедура враћа листу свих хипервеза\index{HTML@\emph{HTML}!hiperlink@хиперлинк} са дате веб странице.

\subsection{Завршетак веб-паука}

Веб паук би требао да нађе све хипервезе на задатој веб страници и да их смести
у листу. Даље, веб-паук наставља процес следећи пронађене хипервезе и на тим веб
страницама ће налазити нове хипервезе. Да се процес тражења хипервеза не би више
пута понављао на истим веб страницама, потребно је користити две променљиве:

\begin{description}
\item[tocrawl] у овој листи ће бити смештене странице које је потребно прегледати
\item[crawled] у овој листи ће бити смештене странице које су већ прегледане
\end{description}

\subsubsection{Завршни к\^oд}

Претпоставља се да су познате процедуре \textbf{get\_next\_target} и \textbf{get\_all\_pages}, чији к\^{о}д је наведен у претходним поглављима(в. к\^{о}дове \ref{lst:get_next_target_2}, \ref{lst:print_all_links}). Да би се к\^{о}д нове процедуре \textbf{crawl\_web} учинио прегледнијим, биће уведена и процедура \textbf{union} која прави унију две листе. Процедура \textbf{get\_page}  узима хипервезу и враћа HTML к\^{о}д те стране. Ако се страница не може отворити из разних разлога и/или је дата веб страница празна, процедура враћа празну ниску.

\begin{lstlisting}[caption=Веб паук, label={lst:crawlweb1}, numbers=left]
def get_page(url):
    try:
        import urllib
        return urllib.urlopen(url).read()
    except:
        return ""

def union(a, b):
    for e in b:
        if e not in p:
            p.append(e)

def crawl_web(seed):
    tocrawl = [seed]
    crawled = []
    while tocrawl:
        page = tocrawl.pop()
        if page not in crawled:
            union(tocrawl, get_all_links(get_page(page)))
            crawled.append(page)
    return crawled
\end{lstlisting}

Процедура прво поставља две листе, оно што је потребно претражити и ту поставља први елемент, иницијалну страницу. Друга променљива је она у коју се смештају странице које су већ претражене и она је иницијално празна. Кад се страница претражи, она се додаје у \textbf{crawled} листу да се не би више пута непотребно претраживала.

У линији 9 почиње петља, која ради док листа \textbf{tocrawl} има елемената. Кад листа више нема страница за скенирање, процедура враћа листу прегледаних страна. У петљи се скида последња хипервеза из \textbf{tocrawl} листе и на основу одговора после питања из услова да ли је та страница већ претражена, процедура је даље претражује или не. На крају хипервеза бива додата листи већ претражених страна.

\subsubsection{Ограничен број страна по свакој хипервези}

Уколико не постоји потреба за претраживањем свих веб страница на које упућују
хипервезе које су прегледане у процесу тражења свих хипервеза, који је описан у претходном поглављу, већ је потребно прегледати само првих неколико страница, тада да је могуће одредити колико ће се веб страница претражити. У наредном примеру, вредност максималног броја прегледаних страница ће бити смештена у променљиву \textbf{max\_pages}:

\begin{lstlisting}[caption=Претраживање са ограниченим бројем страна, label={lst:crawlweb2}, numbers=left]
def crawl_web(seed, max_pages):
    tocrawl = [seed]
    crawled =[]
    count = 0 # vrednost brojaca inicijalno na 0
    while tocrawl and count < max_pages:
        page = tocrawl.pop()
        if page not in crawled:
            union(tocrawl, get_all_links(get_page(page)))
            crawled.append(page)
            count = count + 1 # iteracija brojaca posle svakog linka
    return crawled
\end{lstlisting}

Дакле, петљa se прекида ако је листа празна или ако је прекорачен максимални број хипервеза по страници.

\subsubsection{Ограничена ,,дубина'' претраживања}

Ефикаснији начин ограничавања рада веб-паука да не претражује све странице је да се постави одговарајућа дубина претраживања. Дубина ће у наредном примеру бити одређена променљивом \textbf{max\_depth}.

\begin{lstlisting}[caption=Скенирање ограничено по дубини, label={lst:crawlweb3}, numbers=left]
def crawl_web (seed, max_depth):
    tocrawl = [seed, 0] # postavlja se 0 kao pocetna dubina
    crawled = []
    while tocrawl:
        layer = tocrawl.pop()
        depth = layer[1]
        if depth <= max_depth:
            for url in layer[0]:
               if url not in crawled:
                 union(tocrawl, [get_all_links(get_page(url)),depth+1])
                 # dubina se povecala za 1
                 crawled.append(url)
    return crawled
\end{lstlisting}

За потребе оваквог ограничавања, измењена је листа \textbf{tocrawl}, у којој
сада постоји пар елемената: страница и број. Број се повећава кад се прегледају
све хипервезе са одређене странице и пређе на нову страну. Тако се  ограничава
веб-паук да би после одређеног броја прегледаних страница заустави свој рад.
\pagebreak
