\subsection{Рачунање и променљиве}

У даљем тексту се претпоставља да је Python инсталиран на рачунару и апстрахује се начин на који се добија резултат на потенцијалном екрану или принтеру. Дакле, описује се искључиво програмски језик Python, његова синтакса, као и шта се све може да урадити помоћу њега.

\subsubsection{Python као калкулатор}

Python\index{Python!kalkulator@калкулатор} може да послужи и као обичан калкулатор\cite{van2003introduction,lutz2009learning}. Ако је потребно да се изврше основне аритметичке операције: сабирање, одузимање, множење и дељење, треба користити адекватне симболе за дате операције (редом): +, -, *, /. Поставља се питање како се Python односи према разним типовима бројева. Бројеви у рачунарству се обично деле на целобројне и бројеве у покретном зарезу. За операције сабирања, одузимања и множења правило је врло једноставно\index{Python!operacije@операције са бројевима}: ако се ради са целим бројевима, добија се и целобројни резултат. Исто важи и за бројеве у покретном зарезу. Мало је другачији случај са дељењем. Ако се врши дељење целих бројева, количник ће бити дат као децимални број. Ако треба да се као резултат добије цео део количника, користи се симбол ,,//''. На пример:

\begin{lstlisting}[caption = Примери операција са бројевима, label = racun]
>>> 100 + 16
116
>>> 96.0 - 11
85.0
>>> 11.1 * 10
111.0
>>> 20/5
4.0
>>> 20 / 8
2.5
>>> 20 // 8
2
\end{lstlisting}

Moгу се користити и друге математичке операције, као што су степеновање\index{Python!stepenovanje@степеновање} или се може направити програм за произвољну математичку функцију. Степеновање је другачије него што је то у осталим програмским језицима и обавља се помоћу симбола ,,**'':

\begin{lstlisting}[caption = Степеновање, label = stepen]
>>> 2 ** 32
4294967296
\end{lstlisting}

Такође, Python омогућава и рачунање са комплексним бројевима у алгебарском облику\index{Python!kompleksni@комплексни бројеви}, где се код комплексних бројева користи суфикс ,,j'' или ,,J'' ако треба да се назначи имагинарни део тог комплексног броја. На пример:

\begin{lstlisting}[caption = Операције са комплексним бројевима, label = kompleksni]
>>> (1+1j)*(2-1j)
(3+1j)

>>> 1j ** 2
(-1+0j)
>>> 1j * 1J
(-1+0j)
>>>
\end{lstlisting}

\subsubsection{Променљиве}

Променљивој се додељује одговарајућа вредност, која може бити бројчана или логичка (тачно или нетачно) или променљива може да реферише на вредност друге променљиве. Ако је ово последње случај, промена вредности једне променљиве, утиче и на вредност друге променљиве. У примеру који следи, наредбом:

\begin{lstlisting}
>>>a = 1
\end{lstlisting}

\begin{figure}[here]
\centering
\includegraphics{1a.png}
\caption{\emph{а} реферише на вредност 1}
\label{slike:a_je_1}
\end{figure}

Променљива \emph{а} има вредност 1. Може се рећи и да променљива реферише на вредност 1 која се налази негде у меморији.

Ако се истој променљивој додели нека друга вредност, на пример 2, онда ће она реферисати на број 2, док је број 1 напуштен.

\begin{lstlisting}
>>>a = 2
\end{lstlisting}

\begin{figure}[here]
\centering
\includegraphics{2a.png}
\includegraphics{1.png}
\caption{\emph{а} сада реферише на вредност 2}
\label{slike:a_je_2}
\end{figure}

Потом може да се уведе нова променљива и да се подеси да реферише на променљиву \emph{а}.

\begin{lstlisting}
>>>b = a
\end{lstlisting}

\begin{figure} [here]
\centering
\includegraphics{2ab.png}
\caption{и \emph{а} и \emph{b}реферишу на исту вредност}
\label{slike:a_b_je_2}
\end{figure}

Тада и једна и друга променљива реферишу на исту вредност. Ако би се једној од њих променила вредност, тада би друга реферисала на нову вредност.

Имена променљиве се по договору пишу свим малим или свим великим словима латинице, иако Python дозвољава све начине писања променљивих, осим што нису дозвољене кључне речи. Прави Пајтонисти\cite{pythonista} кажу да им је само име ,,променљива'' неприкладно за оно што она ради у Python-у, те да би бољи избор био употребити речи: ,,именa'',  ,,објекти'' или ,,везивања'' (\emph{енгл., bindings}).

Променљиве се могу користити и у рачунању. На пример:

\begin{lstlisting}[caption = Пример коришћења променљивих, label = variables]
>>> br_stanovnika = 9860000 # br. stanovnika Srbije prema proceni iz 2010.
>>> povrsina = 77474 # povrsina Srbije u kv. kilometrima
>>> gustina_naseljenosti = br_stanovnika / povrsina
>>> print (gustina_naseljenosti)
127.26850298164545
\end{lstlisting}

Дакле, могло би се рећи да програмски језик Python садржи функционалност моћног калкулатора који допушта и задавање одговарајуће вредности променљивима (именима) уз поштовање претходно наведеним правила.
