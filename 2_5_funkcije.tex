\subsection{Функције и модули}

Ако се довољно често употребљава одређени кодни блок, који би могао поново да се искористи, онда је пожељно правити функцију. Организација кода у којој се крупне целине деле на мање и смештају у функције је пожељна, јер омогућава бољу прегледност кода, а тиме и лакше проналажење грешака. Такође, у том случају је могуће лакше мењати к\^{о}д, прилагођавати га другим апликацијама, итд.

\subsubsection{Функције у Python-у}

Да би се употребила функција \emph{func}\index{Python!funkcije@функције}, која има улазне параметре, на пример $a_{1}, a_{2},\dots, a_{n}$, она мора да се позове у коду са:

\begin{lstlisting}
>>>func(a1, a2, ..., an)
\end{lstlisting}

Дефинисање функције почиње кључном речи $def$\index{Python!def@\emph{def}}, иза које се наводи назив и параметри функције. Функција не мора имати параметре, а ако их има, онда их има коначно много. На крају реда се поставља специјални знак ,,:'' - двотачка, која у Python-у има улогу да упозори интерпретатор да иза ње следи кодни блок. У следећи ред кода, увученом за 4 празна карактера, започиње кодни блок дате функције.

\begin{lstlisting}[caption = Дефинисање функције, label = func]
>>>def min(a, b):
      if a < b:
          return a
      else:
          return b
\end{lstlisting}

Да би функција вратила неку вредност, потребно је да у дефиницији функције постоји кључна реч $return$\index{Python!return@\emph{return}} после које се наводи вредност која ће бити резултат функције.

\subsubsection{Модули}

Модули\index{Python!modul@модули} служе за груписање функција, променљивих и других структура у веће и моћније програме. Неки модули су уграђени у сам Python: на пример, $tkinter$\footnote{$tkinter$ служи за прављење игара у Python-у}. Такође, постоји могућност учитавања модула које су други поставили на интернет, а који помажу лакшем и бржем писању жељеног кода. Неки од познатијих таквих модула су $PIL$\footnote{PIL више није актуелан од верзије Pythona 2.6. Наследник овог модула је $PILLOW$}  (Python Imaging Library), Panda3D, итд.

Да би се користио модул, потребно је на почетку програма употребити кључну реч $import$, после које се наводи име модула. На пример: $import$ \emph{time}.
