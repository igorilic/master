\subsection{Класе и објекти}

Модеран програмски језик не може се замислити без подршке објектно-ориијентисаном програмирању\footnote{краће \emph{ООП}}. Python, наравно, омогућава коришћење објектно-оријентисаног програмирања и даје му пуну подршку.

\subsubsection{Објектно-оријентисано програмирање}

Овај рад нема амбицију да се бави Објектно-оријентисаним програмињем(у даљем тексту: ООП). У даљем тексту ће се у краћим цртама навести значај, особине и предности ООП.

Појам Објектно-оријентисаног програмирања први пут се помиње почетком '70-их година прошлог века, приликом представљања програмског језика $Smalltalk$ (први језик који је у себи имао елементе ООП је био $Simula67$). Касније се највише везује за програмски језик Java\index{Python!Java@\emph{Java}}, да би се данас широко примењивао у програмирању.

Објектно-оријентисано програмирање је засновано на концепту објекта. \emph{Објекти} су структуре података са придруженим скупом процедура и функција које се називају \emph{методи} и служе за рад са подацима који припадају објекту. Уобичајено је да су методи једини начин за рад са објектима. Сваки објекат је примерак или инстанца неке \emph{класе}. У класи се дефинише садржај објеката те класе и скуп метода који ће омогућити рад са објектима.

Готово у свим објектно-оријентисаним програмским језицима постоји \emph{наслеђивање} као механизам за креирање нових класа из већ постојећих. На тај начин се добијају наткласе и поткласе.

Уобичајена је могућност у објектно-оријентисаном програмирању дефинисање и поткласа класе, која је у односу на своју супер-класу \emph{дете}, док је супер-класа \emph{родитељ}. Поткласа наслеђује све садржаје објеката класе и методе од наткласе. Типичан пример за овакав начин наслеђивања је класа Животиња, где можемо дефинисати поткласе Сисари, Инсекти, Птице, Рибе, итд. Такође, можемо дефинисати и класу Торбари, која ће бити поткласа Сисара и тај процес наслеђивања можемо продужити коначно много.

\subsubsection{Класе и објекти у Python-у}

Да би се креирала класа у Python-у, потребно је користити кључну реч $class$, после које се наводи име класе и евентуално класу наткласе, које је та класа наследила. На пример:

\begin{lstlisting}[caption=Дефинисање класа, label=class]
>>>class Zivotinje:
      pass
>>>class Sisari(Zivotinje):
      pass
\end{lstlisting}

Кључна реч $pass$\index{Python!pass@\emph{pass}} значи да та класа не садржи никакве особине објеката нити методе.

Када постоји потреба за увођењем конкретног примерка класе, тј. објекта\index{Python!objekti@објекти} неке класе, тада је потребно само да навести име објекта и после знака једнакости навести класу и особине које припадају датом објекту, као у следећем примеру:

\begin{lstlisting}[caption=Креирање објекта, label=objects]
>>>class Macke(Sisari):
      pass
>>>tosa = Macke()
\end{lstlisting}

У Python-у не постоје конструктори и деструктори као у неким другим објектно- оријентисаним програмским језицима. Довољно је навести име класе и инстанца је препозната код интерпретатора.

\subsubsection{Методи}

У класи је могуће креирати скуп функција\index{Python!metodi@методи}. Такве функције се називају методима дате класе. Креирање метода се врши дефинисањем функције унутар класе.

Објекту се додељује метод, тако што се после имена објекта поставља тачка и наводи име метода са аргументима, набројаним унутар заграде. Метод не мора да има аргументе.

\begin{lstlisting}[caption= Методи класе, label=method]
>>>class Macke(Sisari):
       def predenje():
           print('purrrrr')
>>>tosa.predenje()
>>>purrrrr
\end{lstlisting}
