\section{Програмски језик Python}

\subsection{Карактеристике и историјат}

\subsubsection{Програмски језици и програмирање}

Програмирање подразумева активност \emph{програмера} усмерену ка решавању конкретног проблема на рачунару. Проблем се обично решава конструкцијом \emph{алгоритма}\index{algoritam@алгоритам} за решење проблема, а алгоритам се записује посредством програмског језика. Програмски језик има улогу да обезбеди конструкције и начине за израчунавање на рачунару. Организовано израчунавање обично називамо \emph{програмом}.

Програмски језик\index{Programski jezik@Програмски језик} је средство за писање програма и саопштавање програма рачунару. То је вештачки језик који првенствено служи за комуникацију између човека и рачунара, мада се понекад програмски језик користи и за комуникацију између људи.

До сада је направљено неколико хиљада програмских језика и потребно је да се на неки начин разврстају. Постоје разне класификације, зависно од \emph{критеријума класификације}:

\begin{itemize}
\item степен зависности од рачунара
\item време настанка и својства
\item област примене
\item начин решавања проблема
\end{itemize}

Тако по степену зависности од рачунара програмске језике можемо поделити на машински зависне (Асемблерски језици, Макро-језици, итд.) и машински независне програмске језике.

По областима примене програмске језике можемо поделити на оне за учење програмирања (Pascal, Basic, Logo, итд.), програмске језике за развој системског софтвера (C, Asembler, итд.), за пословну примену (COBOL, SQL, ...), за развој програма на интернету (Java, Java Script, Perl, итд.), за примену у математици, и друго.

Програмске језике делимо и по начину решавања проблема и то на процедуралне и непроцедуралне. Док бисмо их по времену настанка и својствима поделили на језике I генерације (машински и асемблерски), II генерације (FORTRAN, COBOL, LISP, BASIC,...), III генерације (Pascal, C, PROLOG, Smalltalk, C++, Java, ...) и језике IV генерације (SQL, VisualBasic, сви језици унутар апликација)  \cite{tosic}.

Може се поставити и питање \emph{зашто} учити компјутерско програмирање? Програмирање развија креативност, логичко размишљање и способност решавања проблема. Програмер добија прилику да креира \emph{нешто} из \emph{ничега}, користи логику да претвори скупове речи у компјутерски програм и кад нешто не иде како треба на дело долази способност да се реши проблем и да се нађе и исправи грешка.

\subsubsection{Зашто Python?}
Чињеница је да постоје на хиљаде програмских језика. Неки од њих су поменути у претходном делу. Шта то издваја програмски језик Python\index{Python} од других и зашто би се неко одлучио да програмира у Python-у, уместо, на пример, у Pascal-у?

Python је лак за учење, а уз то има заиста корисна својства за програмера почетника. К\^{о}д је врло разумљив за читање, ако се упореди са осталим програмским језицима. Такође, постоје разни додаци (модули) који ће проширити Python на захтевани начин, тако да Python може да послужи за креирање једноставних анимација са додатком Turtle (инспирисан Turtle graphics, коришћеним у програмском језику Logo у шездесетим годинама прошлог века) \cite{briggs2012python}. Други модули служе да користимо Python као скрипт језик или као додатак разним комплекснијим језицима попут  C/C++.

Надаље, Python може представљати сјајно оруђе за учење основних појмова програмирања, типова података, услова, петљи, за учење разних алгоритама(на пример сортирања), онда за имплементацију математичких формула и начина на који се те формуле могу програмирати (итеративно или рекурзивно, на пример за факторијел).

Важан фактор је такође и \emph{доступност} Python-a, јер се интерпретатор може врло једноставно преузети са интернета и инсталирати на рачунару. Постоје и разне варијанте Python интерпретатора које се налазе на појединим веб страницама. Python може да се користи на многобројним оперативним системима: Windows-у, Mac OS-у, Linux-у, па и на мобилним платформама iOS-у или на Android оперативном систему  \cite{pythonsite}.

\subsubsection{Историјат и начин коришћења}
Програмски језик Python\index{Python} је настао почетком '90-их година прошлог века. Креирао га је Холанђанин Гвидо ван Росум (Guido van Rossum)\index{Python!Guido van Rossum} и наставио да га развија до данашњег дана, наравно уз помоћ огромне заједнице програмера широм света. Данас је актуелна верзија 3.4. Гвидо је у свету познат под надимком BDFL, што је скраћеница за \emph{Доживотни Добронамерни Диктатор}\footnote{Benevolent Dictator For Life}.

Python је добио име по култној британској хумористичкој серији у продукцији BBC-а, ,,Летећи циркус Монти Пајтона'' (\emph{Monthy Python's Flying Circus}), коју су прославили легендарни глумци и комичари попут: Ерика Ајдла, Џона Клиза, Мајкла Пелина, Грејема Чепмена, режисера Терија Гилијама и других. Снимљена као алтернативни театар апсурда, серија је остала позната и у данашње време и до данас се памте и радо гледају скечеви попут ,,\emph{Министарства смешног ходања}'', ,,\emph{Најсмешнији виц на свету}'', итд.

Оно што је још остало je да се опише на који начин може да се користи Python на рачунару. Дакле, потребан је Python интерпретатор, који се може бесплатно преузети са званичне Python-ове веб презентације \cite{pythonsite}, као и неки од текст едитора и/или развојних окружења (нпр. \emph{Eclipse}, \emph{PyCharm}, \emph{Vim} или било који текст едитор) помоћу кога ће се уносити к\^{о}д. Алтернативно, може се преузети и инсталирати апликација \emph{IDLE}\index{Python!IDLE@\emph{IDLE}} (\texttt{http://www.python.org/getit/} ), која омогућује да се програмира у Python-у на врло једноставан начин.
