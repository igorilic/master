\section{Додатак Б: Резултати теста хеш-табела}\label{sec:dodatakb}

Тестирање хеш-табеле се извршило како би се дошло до закључка који начин
креирања хеш-табеле је бољи: коришћењем листа или мапа. Да би се дошло до
резултата коришћени су програмски кодови које наводимо у потпуности у коду \ref{lst:listtest} и \ref{lst:dicttest}.
За потребе тестирања употребљена је изворна страница \url{http://poincare.matf.bg.ac.rs/~vladaf/index_e.html}
којом веб-паук започиње свој рад.
Тестирана је употреба хеш-табела за 10, 20 или 50 максималних
веб-страница, са којих веб-паук прикупља податке. Резултати су дати у табели \ref{tabele:list-dict}
у секундама, коришћењем хронометра из Python-овог додатка \emph{time}.

\lstset{numbers=left}
\lstinputlisting[caption=Тестирање хеш-табеле са листама, label={lst:listtest}]{list-test.py}

\lstset{numbers=left}
\lstinputlisting[caption=Тестирање хеш-табеле са мапом, label={lst:dicttest}]{dict-test.py}

\begin{table}[h]
\centering
\begin{tabular}{|l|r|r|r|} \hline
\textbf{Broj strana} & 10 & 20 & 50\\ \hline
\textbf{Lista} & 0,080776 & 0,381066 & 322,546326\\ \hline
\textbf{Mapa} & 0,022313 & 0,046528 & 0,523919\\ \hline
\end{tabular}
\caption{Тестирања рада веб-паукова са листама и мапама}
\label{tabele:list-dict}
\end{table}

Закључује се да веб-паук који ради који користи мапу, много брже прикупља
податке од веб-паука који ради са хеш-табелом која користи искључиво листе.
Штавише, што је већи број страница са којих се прикупљају подаци,
уочава се већа разлика у брзини рада.
