\section{Додатак Б: Резултати}\label{sec:dodatakb}
		Да би се добила слика о значају рангирања страница, направљен је модел четири повезане веб странице, чији је граф дат на слици \ref{slike:graf}. К\^{о}дови страница изгледају овако:\\
		
\begin{lstlisting}[language=HTML, caption=K\^{о}д прве странице, label={lst:prva}, numbers=left]
<html>
<head>
 <title>Prva stranica</title>
 <link rel="stylesheet" href="my.css">
</head>
<body>
 <h1>Prva stranica</h1>
 <ul>
  <li><a href="http://localhost/druga.html">Druga stranica</a></li>
  <li><a href="http://localhost/treca.html">Treca stranica</a></li>
  <li><a href="http://localhost/cetvrta.html">Cetvrta stranica</a></li>
 </ul>
 <p>Prva stranica ima hipervezu ka drugoj stranici, trecoj stranici,
 cetvrtoj stranici. Ova stranica se realizuje u cilju testiranja veb
 pretrazivaca.</p>
</body>
</html>
\end{lstlisting}%
\medskip

\begin{lstlisting}[language=HTML, caption=K\^{о}д друге странице, label={lst:druga}, numbers=left]
<html>
<head>
 <title>Druga stranica</title>
 <link rel="stylesheet" href="my.css">
</head>
<body>
 <h1>Prva stranica</h1>
 <ul>
  <li><a href="http://localhost/prva.html">Prva stranica</a></li>
 </ul>
 <p>Druga stranica ima hipervezu ka prvoj stranici. Ova stranica se realizuje u cilju testiranja veb
 pretrazivaca.</p>
</body>
</html>
\end{lstlisting}%
\medskip
\pagebreak
\begin{lstlisting}[language=HTML, caption=K\^{о}д треће странице, label={lst:treca}, numbers=left]
<html>
<head>
 <title>Treca stranica</title>
 <link rel="stylesheet" href="my.css">
</head>
<body>
 <h1>Treca stranica</h1>
 <ul>
  <li><a href="http://localhost/prva.html">Cetvrta stranica</a></li>
 </ul>
 <p>Treca stranica ima hipervezu ka cetvrtoj stranici. Ova stranica se realizuje u cilju testiranja veb
 pretrazivaca.</p>
</body>
</html>
\end{lstlisting}%
\medskip
\begin{lstlisting}[language=HTML, caption=K\^{о}д четврте странице, label={lst:prva}, numbers=left]
<html>
<head>
 <title>Cetvrta stranica</title>
 <link rel="stylesheet" href="my.css">
</head>
<body>
 <h1>Cetvrta stranica</h1>
 <ul>
  <li><a href="http://localhost/prva.html">Prva stranica</a></li>
 </ul>
 <p>Cetvrta stranica ima hipervezu ka prvoj stranici. Ova stranica se realizuje u cilju testiranja veb
 pretrazivaca.</p>
</body>
</html>
\end{lstlisting}%
\medskip
Странице су затим постављене на локални сервер. Кад се уз к\^{о}д \ref{lst:finalcode} пусти наредба:\\
\lstinline{print(lookup(index, 'stranica')}, добија се следећи резултат:
\begin{lstlisting}
['http://localhost/prva.html',
 'http://localhost/prva.html',
 'http://localhost/cetvrta.html',
 'http://localhost/cetvrta.html',
 'http://localhost/treca.html',
 'http://localhost/treca.html',
 'http://localhost/druga.html',
 'http://localhost/druga.html']
\end{lstlisting}
Очигледно је да обично прегледање без рангирања даје списак свих хипервеза који одговарају датој кључној речи.
\pagebreak
Сада ће уз исти к\^{о}д бити дата наредба:\\
\lstinline{print(compute_ranks(graph))} и тада се добија
\begin{lstlisting}
{'http://localhost/treca.html': 0.1572993996378601,
 'http://localhost/cetvrta.html': 0.28313891934814817,
 'http://localhost/prva.html': 0.40226228137613174,
 'http://localhost/druga.html': 0.1572993996378601}
\end{lstlisting}
У овом случају, испоставља се мапа у којој су кључеви хипервезе, а вредности њихов ранг у моделу.\\
Ако је потребно поређати хипервезе по редоследу рангирања, онда се наредбом:\\
\lstinline{print(rank_list(compute_graph(graph)))}, добија листа\\
\begin{lstlisting}
['http://localhost/prva.html',
 'http://localhost/cetvrta.html',
 'http://localhost/treca.html',
 'http://localhost/druga.html']
\end{lstlisting}
И за сам крај, могуће је доставити кориснику тачно једну хипервезу, која одговара упиту и која има највећи ранг од свих хипервеза које одговарају упиту.\\
\lstinline{print(lucky_search(index, compute_ranks(graph), 'stranica'))}, тако да је резултат само прва страница.
\begin{lstlisting}
http://localhost/prva.html
\end{lstlisting}

Може се извући закључак да је далеко корисније претраживање са укљученим модулом рангирања, које омогућава кориснику да за свој задати упит добије као резултат једну или више рангираних страница и тиме добије адекватну информацију.