\section{Програмски језик Python}
		\subsubsection{Програмски језици и програмирање}
		Програмирање подразумева активност \emph{програмера} усмерена ка решавању конкретног проблема на рачунару. Проблем се обично решава конструкцијом \emph{алгоритма}\index{algoritam@алгоритам} за решење проблема, а алгоритам се записује посредством програмског језика. Програмски језик има улогу да обезбеди конструкције и начине за израчунавање на рачунару. Организовано израчунавање обично називамо \emph{програмом}.\\
		Програмски језик\index{Programski jezik@Програмски језик} је средство за писање програма и саопштавање програма рачунару. То је вештачки језик који првенствено служи за комуникацију између човека и рачунара, мада се понекад програмски језик користи и за комуникацију између људи.\\
		 		До сада је направљено неколико хиљада програмских језика и потребно је да се на неки начин разврстају. Постоје разне класификације, зависно од \emph{критеријума класификације}:
		\begin{itemize}
		\item степен зависности од рачунара
		\item време настанка и својства
		\item област примене
		\item начин решавања проблема
		\end{itemize}
		Тако по степену зависности од рачунара програмске језике можемо поделити на машински зависне (Асемблерски језици, Макро-језици, итд.) и машински независне програмске језике.\\
		По областима примене програмске језике можемо поделити на оне за учење програмирања (Pascal, Basic, Logo, итд.), програмске језике за развој системског софтвера (C, Asembler, итд.), за пословну примену (COBOL, SQL, ...), за развој програма на интернету (Java, Java Script, Perl, итд.), за примену у математици, и друго.\\
		Програмске језике делимо и по начину решавања проблема и то на процедуралне и непроцедуралне. Док би их по времену настанка и својствима поделили на језике I генерације (машински и асемблерски), II генерације (FORTRAN, COBOL, LISP, BASIC,...), III генерације (Pascal, C, PROLOG, Smalltalk, C++, Java, ...) и језике IV генерације (SQL, VisualBasic, сви језици унутар апликација).\cite{tosic}\\
		Може се поставити и питање \emph{зашто} учити компјутерско програмирање? Програмирање развија креативност, логичко размишљање и способност решавања проблема. Програмер добија прилику да креира \emph{нешто} из \emph{ничега}, користи логику да претвори скупове речи у компјутерски програм и кад нешто не иде како треба на дело долази способност да се реши проблем и да се нађе и исправи грешка. \\
		Програмирање је и забава. Понекад је то и захтевна (и с времена на време фрустрирајућа) активност, али где се те способности могу искористити како у школи и на послу, тако и у кући невезано за захтеве каријере и образовања.\\