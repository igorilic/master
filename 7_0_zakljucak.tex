\section{Закључак}

Програмски језик Python поседује разноврстан избор типова података, структура, омогућава програмеру да у кратком временском периоду направи моћан програм. Python поседује све што модеран програмски језик мора да има. Но, Python поседује  и додатни квалитет: армију корисника који учествују у даљем развоју језика својим саветима и расправама на интернет форумима и групама. Python је веома ,,жив'' програмски језик, који периодично избацује нове верзије и побољшава перформансе.

Додаци које Python доноси са својим модулима и проширењима чине га конкурентним у односу на комерцијалне програмске језике какви су Java, C\#,... Може се користити и као скрипт језик у оквиру прављења веб апликација, може се користити за писање мањих делова програма у C/C++, а може се користити и самостално за писање десктоп и веб апликација. Данас је Python и незаменљиво оруђе у рукама научника, који га користе за компликоване прорачуне, анализу и презентацију података.

Проширење у виду $django$\index{Python!django@\emph{django}} \emph{framework}-а\cite{django} омогућава широку употребу Python-а у писању веб апликација, пре свега CMS\footnote{CMS (Content Management System) je програмска апликација која омогућава креирање, модификовање, брисање и организовање садржаја преко централног интерфејса. Често се користи за веб апликације које садрже блогове, вести, омогућавају куповину преко Интернета.}, али и мањих веб страница.

Програмски језик Python у овом раду је у потпуности одговорио на потребе писања веб претраживача. Током писања к\^{о}да, коришћене су само основне функције Python-а. Нису се употребљавале компликоване и робустне структуре, што омогућава потенцијалном читаоцу лак увид у к\^{о}д и процену шта ће к\^{о}д на крају даје као резултат.

Процес који је описан на почетку поглавља \ref{subsec:web} успешно је окончан. Оно што превазилази тему овог рада је даља имплементација веб претраживача у веб апликацији. Међутим, може се закључити да је претраживање са укљученим модулом рангирања далеко подесније, него ли оно које само испоставља списак линкова без претходног рангирања (в. поглавље \ref{sec:dodatakb}).

Веб претраживање данас је уносан посао. Компаније које нуде услуге веб претраживања наплаћују другим фирмама за резултате својих моћних веб-паукова, као и за трошкове рекламирања, на пример. У модерном добу, у ком технолошки развој оставља штампане енциклопедије на ропотарницу историје, неопходно је имати тачан и поуздан систем веб претраживања. Такође, резултати претраживања сугеришу компанијама на који начин ће лакше доћи до купца. У овом тренутку расте потреба за таквим видом информација. Процес таквог прикупљања информација, назива се \emph{data mining}(енгл., \emph{истраживање података}), где се у процесу сакупљања огромне количине података са Интернета, покушавају донети закључци у циљу бољег функционисања компаније.

Дакле, може се закључити да се предлаже примена језика Python за реализацију алгоритама којим се врши рангирање веб страница. На крају је са успехом имплементиран Python програм за претраживање веба и рангирање страница.
