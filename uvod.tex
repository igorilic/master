\section{Увод}

Tоком развоја Интернета и поготово од настанка веба, повећавала се потреба за претраживањем страница које су биле постављене на Интернет. Са настанком веба и његовим ширењем та потреба се продубила због огромног броја информација које су преплавиле веб од настанка до данас. Највећи проблем са којим су се корисници Интернета, да се из мора информација издвоји она која је најподеснија за корисников упит. Овај рад покушаће да реши проблем претраживања  веб страница, кроз реализацију алгоритма за рангирање веб страница у програмском језику Python. Такође, у току тог процеса, биће анализиран алгоритам за рангирање PageRank\texttrademark\ \footnote{PageRank\texttrademark\ алгоритам није јавно доступан, те ће се његова анализа свести на оно што јесте јавно доступно и на анализе других аутора, нпр. \cite{langville2011google}  }  и његове предности у односу на претраживање без примене датог алгоритма.

Рад ће бити подељен на неколико делова. У првом делу биће укратко објашњена синтакса и рад програмског језика Python. Обратиће се посебна пажња на основне елементе Python-а, као и на начин писања програма у њему. Такође, уз навођење једноставних примера, биће обрађени и типови података који се користе у Python-у, начин на који се уводе услови, петље, функције и модули, као и на који начин Python приступа класама и објектима.

После дела о програмском језику Python биће приложен део о претраживању Интернета и обради хипервеза. Биће приложен и пример како се из HTML к\^{о}да издвајају хипервезе, што ће чинити основу веб-паука.

Затим се описује реализација веб-паука, који скупља све хипервезе на веб страници и потом следи те хипервезе да би наставио свој процес унедоглед или до дубине претраживања који му се одреди.

После тога долази се до процесирања корисниковог упита и описује се како ће претраживач одговарати на упите. Поставља се питање да ли је процесирање могуће радити на бржи начин и предлаже се коришћење хеш табеле.

На крају, рад се бави питањем рангирања веб страница у процесу претраживања. Анализира се алгоритам PageRank\texttrademark\ даје се могуће решење овог алгоритма у програмском језику Python.

После закључка, дати су и додаци који садрже комплетан к\^{о}д, затим поређење резултата приликом задавања модела претраживања са и без укљученог модула за рангирање страница. Такође, дат је списак програмских к\^{о}дова који су се појавили током писања овог рада, као и литература која је коришћена током писања рада.

У примерима који се налазе у мастер раду је коришћена верзија Python-а 2.7.6, мада би примери требали да раде и са новим верзијама. Примери су тестирани у програмском окружењу које чини едитор \emph{Vim}, уз одговарајуће додатке\cite{vim}, а у оквиру оперативног система \emph{Ubuntu Linux 14.04}.


